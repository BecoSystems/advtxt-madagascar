%%%%%%%%%%%%%%%%%%%%%%%%%%%%%%%%%%%%%%%%%%%%%%%%%%%%%%%%%%%%%%%%%%%%%%%%%%%%%%%%%%%%%%%%
% Criação de Fluxograma usando LaTeX
%
% Assunto: escrever aqui um comentário com uma
%          breve explicação do exercício
%
% Autores:
%     Nome do Aluno 1
%     Nome do Aluno 2
%     Nome do Aluno 3
%
% Coordenação:
%     Prof. Dr. Ruben Carlo Benante
%
% Data: 2024-04-25
%%%%%%%%%%%%%%%%%%%%%%%%%%%%%%%%%%%%%%%%%%%%%%%%%%%%%%%%%%%%%%%%%%%%%%%%%%%%%%%%%%%%%%%%

%%%%%%%%%%%%%%%%%%%%%%%%%%%%%%%%%%%%%%%%%%%%%%%%%%%%%%%%%%%%%%%%%%%%%%%%%%%%%%%%%%%%%%%%
% Para gerar o PDF use o comando make com o makefile configurado:
%
%    $ make ext-programa2-benante-sobrenome1-sobrenome2.pdf
%
% O conteúdo do makefile é composto dos 3 seguintes comandos que ficam assim automatizados:
%    $ pdflatex exN-fluxograma.tex -o exN-fluxograma.pdf
%    $ bibtex biblio
%    $ pdflatex exN-fluxograma.tex -o exN-fluxograma.pdf

%%%%%%%%%%%%%%%%%%%%%%%%%%%%%%%%%%%%%%%%%%%%%%%%%%%%%%%%%%%%%%%%%%%%%%%%%%%%%%%%%%%%%%%%
% preambulo %%%%%%%%%%%%%%%%%%%%%%%%%%%%%%%%%%%%%%%%%%%%%%%%%%%%%%%%%%%%%%%%%%%%%%%%%%%%
\documentclass[a4paper,12pt]{article} %twocolumn
\usepackage[left=2.5cm,right=2cm,top=2.5cm,bottom=2cm]{geometry}
\usepackage[utf8]{inputenc} % letras acentuadas
\usepackage[portuguese]{babel} % tradução de títulos
\usepackage[colorlinks]{hyperref}
\usepackage{tikz} % para adicionar fluxogramas
\usepackage{algorithm} % ambiente para índice de algoritmos
\usepackage{algpseudocode} % fonte e estilo do algoritmo
\usepackage{graphicx} % permite adicionar imagens
\usepackage{indentfirst} % indenta o primeiro parágrafo também
\usepackage{url} % permite adicionar links de URLs e emails
% \usepackage{natbib}
%[noend]

\DeclareUrlCommand\email{\urlstyle{mm}} % comando para email bonito
\floatname{algorithm}{Algoritmo} % tradução da palavra algoritimo no ambiente de índice

\usetikzlibrary{shapes.geometric, shapes.symbols,arrows} % ajuste do tikz para incluir formas e setas

%%%%%%%%%%%%%%%%%%%%%%%%%%%%%%%%%%%%%%%%%%%%%%%%%%%%%%%%%%%%%%%%%%%%%%%%%%%%%%%%%%%%%%%%
% capa %%%%%%%%%%%%%%%%%%%%%%%%%%%%%%%%%%%%%%%%%%%%%%%%%%%%%%%%%%%%%%%%%%%%%%%%%%%%%%%%%
\title{Fluxograma: nome do fluxograma}
\author{Diego Nascimento Dos Santos \\ Autor2 \\ Autor3}

\begin{document}

\maketitle

%%%%%%%%%%%%%%%%%%%%%%%%%%%%%%%%%%%%%%%%%%%%%%%%%%%%%%%%%%%%%%%%%%%%%%%%%%%%%%%%%%%%%%%%
% definicao dos blocos do fluxograma (tikz) %%%%%%%%%%%%%%%%%%%%%%%%%%%%%%%%%%%%%%%%%%%%

\tikzstyle{line} = [draw, -latex']
\tikzstyle{startend} = [draw, ellipse,fill=red!20, minimum height=2em, node distance=1.55cm]
\tikzstyle{print} = [tape, fill=blue!20, draw, draw=black, minimum width=3cm, minimum height=1.4cm, text width=4.5em, text centered, tape bend top=none, tape bend height=0.2cm, node distance=1.55cm]
\tikzstyle{input} = [trapezium, trapezium left angle=60, trapezium right angle=90, minimum width=3cm, minimum height=1cm, text centered, draw=black, fill=blue!30, node distance=1.95cm]
\tikzstyle{process} = [rectangle, minimum width=3cm, minimum height=1cm, text centered, draw=black, fill=orange!30, node distance=1.55cm]

\tikzstyle{block} = [rectangle, draw, fill=blue!20, text width=5em, text centered, rounded corners, minimum height=4em, node distance=1.55cm]
\tikzstyle{decisionb} = [diamond, draw, fill=blue!20, text width=4.5em, text badly centered, inner sep=0pt, node distance=1.55cm]
\tikzstyle{decision} = [diamond, minimum width=3cm, minimum height=1cm, text centered, draw=black, fill=green!30, node distance=2.25cm]
\tikzstyle{empty} = [circle, fill=white, minimum width=0.01mm, node distance=2.55cm]

%%%%%%%%%%%%%%%%%%%%%%%%%%%%%%%%%%%%%%%%%%%%%%%%%%%%%%%%%%%%%%%%%%%%%%%%%%%%%%%%%%%%%%%%
% resumo %%%%%%%%%%%%%%%%%%%%%%%%%%%%%%%%%%%%%%%%%%%%%%%%%%%%%%%%%%%%%%%%%%%%%%%%%%%%%%%

\begin{abstract}

\textbf{Assunto:} Programa tal tal

% descrever em poucas palavras seu projeto aqui

O programa tal e tal faz isso e isso. Neste artigo iremos apresentar o seu fluxograma completo.
% e (opcionalmente) o seu algoritmo.

Após a modelagem do fluxograma e desenvolvimento da lógica de programação em algoritmo,
o programa será implementado na Linguagem de Programação \texttt{C}

\textbf{Local:} Escola Politécnica de Pernambuco - UPE/POLI

\textbf{Órgão Financiador:} N/A

\textbf{Caracterização:} Modelagem, Projeto e Implementação de Software em Linguagem \texttt{C}

% Este é o fim do resumo.

\end{abstract}

%%%%%%%%%%%%%%%%%%%%%%%%%%%%%%%%%%%%%%%%%%%%%%%%%%%%%%%%%%%%%%%%%%%%%%%%%%%%%%%%%%%%%%%%
% artigo %%%%%%%%%%%%%%%%%%%%%%%%%%%%%%%%%%%%%%%%%%%%%%%%%%%%%%%%%%%%%%%%%%%%%%%%%%%%%%%
% seção de introdução %%%%%%%%%%%%%%%%%%%%%%%%%%%%%%%%%%%%%%%%%%%%%%%%%%%%%%%%%%%%%%%%%%
\section{Introdução}

% Descrever melhor seu projeto aqui

Este programa faz isso, isso e \texttt{também em fonte mono-espaçada faz isso}.

O programa será modelado em \textit{fluxograma} em uma primeira fase, em seguida
sua lógica será desenvolvida em formato de \textit{algoritmo}, para então
na terceira fase ser implementado em Linguagem de Programação \texttt{C}.

%%%%%%%%%%%%%%%%%%%%%%%%%%%%%%%%%%%%%%%%%%%%%%%%%%%%%%%%%%%%%%%%%%%%%%%%%%%%%%%%%%%%%%%%
% seção de objetivos %%%%%%%%%%%%%%%%%%%%%%%%%%%%%%%%%%%%%%%%%%%%%%%%%%%%%%%%%%%%%%%%%%%
\section{Fluxograma}

% adicionar aqui o fluxograma

\begin{tikzpicture}
    % colocar nodos
    \node (inicio) [startend] {Início};
    \node (intro) [print, below of=inicio] {introdução};
    \node (obj) [process, below of=intro] {obj};
    \node (dec1) [decision, below of=obj] {obj == 2};
    \node (txtMorreu1) [print, right of=dec1, node distance=4cm] {texto morreu};
    \node (dec1Sim) [circle, draw, fill=yellow!20, below of=dec1, node distance=2.4cm] {2};
    \node (dec2) [decision, below of=dec1Sim] {vrb == 1};
    \node (txtMorreu2) [print, right of=dec2, node distance=4cm] {texto morreu};
    \node (fim) [startend, below of=dec2, node distance=2.4cm] {Fim};
    \node (dec2Sim) [circle, draw, fill=yellow!20, below of=fim, node distance=2.4cm] {1};
    \node (txtSobreviveu) [print, below of=dec2Sim] {texto usuário escapou e sobreviveu};

    % Desenhar as setas
    \path [line] (inicio) -- (intro);
    \path [line] (intro) -- (obj);
    \path [line] (obj) -- (dec1);
    \path [line] (dec1) -- node[anchor=south] {n} (txtMorreu1);
    \path [line] (dec1) -- node[anchor=east] {s} (dec1Sim);
    \path [line] (dec1Sim) -- (dec2);
    \path [line] (dec2) -- node[anchor=south] {n} (txtMorreu2);
    \path [line] (dec2) -- node[anchor=east] {s} (fim);
    \path [line] (fim) -- (dec2Sim);
    \path [line] (dec2Sim) -- (txtSobreviveu);
\end{tikzpicture}

\clearpage % inicia próxima seção em nova página
%%%%%%%%%%%%%%%%%%%%%%%%%%%%%%%%%%%%%%%%%%%%%%%%%%%%%%%%%%%%%%%%%%%%%%%%%%%%%%%%%%%%%%%%
% seção de justificativa %%%%%%%%%%%%%%%%%%%%%%%%%%%%%%%%%%%%%%%%%%%%%%%%%%%%%%%%%%%%%%%
% \section{Algoritmo}

% adicionar aqui o algoritmo (opcional)

% \clearpage % inicia próxima seção em nova página
%%%%%%%%%%%%%%%%%%%%%%%%%%%%%%%%%%%%%%%%%%%%%%%%%%%%%%%%%%%%%%%%%%%%%%%%%%%%%%%%%%%%%%%%
% Autores %%%%%%%%%%%%%%%%%%%%%%%%%%%%%%%%%%%%%%%%%%%%%%%%%%%%%%%%%%%%%%%%%%%%%%%%%%%%%%
\section*{Detalhamento dos Autores}

%%%%%%%%%%%%%%%%%%%%%%%%%%%%%%%%%%%%%%%%%%%%%%%%%%%%%%%%%%%%%%%%%%%%%%%%%%%%%%%%%%%%%%%%
% Discentes %%%%%%%%%%%%%%%%%%%%%%%%%%%%%%%%%%%%%%%%%%%%%%%%%%%%%%%%%%%%%%%%%%%%%%%%%%%%
\subsection*{Discentes}

\begin{enumerate}
    \item \textbf{Nome Completo:} Fulano de Tal Um
    \begin{description}
        \item [Email:] \email{blabla@poli.br}
        \item [Endereço:]
        \item [Matrícula:]
        \item [CPF:]
        \item [RG:]
        \item [Telefone:]
        \item [Currículo Lattes:] \url{http://lattes.cnpq.br/nnnnn}
    \end{description}

    \item \textbf{Nome Completo:} Fulano de Tal Dois
    \begin{description}
        \item [Email:] \email{blabla@poli.br}
        \item [Endereço:]
        \item [Matrícula:]
        \item [CPF:]
        \item [RG:]
        \item [Telefone:]
        \item [Currículo Lattes:] \url{http://lattes.cnpq.br/nnnnn}
    \end{description}
\end{enumerate}

\end{document}
